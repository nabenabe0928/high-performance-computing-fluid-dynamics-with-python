\documentclass[a4paper,11pt]{report}
\usepackage[T1]{fontenc}
\usepackage[utf8]{inputenc}
\usepackage{lmodern}

\usepackage{amsmath,amssymb,bm,bbm}
\usepackage{graphicx}
\usepackage{here}
% \usepackage{subcaption}
\usepackage{ascmac}
\usepackage{fancyhdr}
\usepackage{algorithm, algpseudocode}
\usepackage{algpseudocode}
\usepackage{tikz}
\usepackage{ulem}
\usepackage{booktabs}
\usepackage{multirow}
\usepackage[caption=false]{subfig}
\usepackage{comment}
\usepackage{listings}
\usetikzlibrary{chains}
\usetikzlibrary{calc}
\usepackage{amsmath,tikz}
\usepackage{lastpage}
\usepackage{tcolorbox}
\usepackage{cancel}
\tcbuselibrary{breakable, skins, theorems}
\newtheorem{theorem1}{Theorem}
\newtheorem{theorem2}{Definition}
\newtheorem{theorem3}{Assumption}
\def\qed{\hfill $\Box$}

% remove the end from algorithm
\algtext*{EndFor}
\algtext*{EndWhile}
\algtext*{EndIf}
\algtext*{EndProcedure}
\algtext*{EndFunction}
\algdef{SE}[SUBALG]{Indent}{EndIndent}{}{\algorithmicend\ }%
\algtext*{Indent}
\algtext*{EndIndent}
\newcommand{\Break}{\textbf{break}}
\newcommand{\Continue}{\textbf{continue}}

\setlength{\textwidth}{160mm}
\setlength{\textheight}{220mm}
\setlength{\oddsidemargin}{-1mm}
\setlength{\voffset}{-15mm}
\setlength{\headsep}{10mm}

\renewcommand{\_}{{\tiny \textunderscore}}
% \renewcommand{\_}{{\fontsize{1pt}{1pt}\selectfont \textunderscore}}

\newcommand{\xv}{\boldsymbol{x}}
\newcommand{\vv}{\boldsymbol{v}}
\newcommand{\uv}{\boldsymbol{u}}
\newcommand{\Uv}{\boldsymbol{U}}
\newcommand{\wv}{\boldsymbol{w}}
\newcommand{\pd}[2]{\frac{\partial #1}{\partial #2}}
\newcommand{\od}[2]{\frac{d #1}{d #2}}
\newcommand{\biggNorm}[1]{\biggl|\biggl| #1 \biggr|\biggr|}
\newcommand{\pdtwo}[2]{\frac{\partial^2 #1}{\partial #2^2}}

\newcommand{\cv}{\boldsymbol{c}}
\newcommand{\feq}{f^{\rm eq}}
\newcommand{\dt}{\Delta t}

\usepackage{hyperref}
\usepackage{graphicx}
\usepackage[english]{babel}

\usepackage{graphicx}
\usepackage{comment}

\usepackage{listings} % package for listing parts of code

\renewcommand*\footnoterule{}

\makeatletter
\renewcommand{\@chapapp}{}% Not necessary...
\newenvironment{chapquote}[2][2em]
  {\setlength{\@tempdima}{#1}%
   \def\chapquote@author{#2}%
   \parshape 1 \@tempdima \dimexpr\textwidth-2\@tempdima\relax%
   \itshape}
  {\par\normalfont\hfill--\ \chapquote@author\hspace*{\@tempdima}\par\bigskip}
\makeatother

% Book's title and subtitle
\title{\Huge \textbf{High Performance Computing with Python} \vspace{4mm} \\ \huge Final Report}
% Author
% \author{\textsc{First-name Last-name}\footnote{email address}}
\author{\textsc{Shuhei Watanabe} \\ \vspace{3mm}\text{5171091}  \\
\vspace{3mm}\text{watanabs@informatik.uni-freiburg.de}}


\begin{document}

\makeatletter
    \begin{titlepage}
        \begin{center}
            \includegraphics[width=0.5\linewidth]{logos/Uni_Logo-Grundversion_E1_A4_CMYK.eps}\\[4ex]
            {\huge \bfseries  \@title }\\[2ex] 
            {\LARGE  \@author}\\[30ex] 
            {\large \@date}
        \end{center}
    \end{titlepage}
\makeatother
\thispagestyle{empty}
\newpage

\tableofcontents

\begin{comment}
    https://publikationen.uni-tuebingen.de/xmlui/bitstream/handle/10900/87663/bwHPC2018-25-Pastewka-Lattice_Boltzmann_with_Python.pdf?sequence=1&isAllowed=y
    Criterion
    o 1. Shear wave decay
        o 1.1. Density evolution plot
        o 1.2. Velocity evolution plot
        o 1.3. Measured viscosity
    o 2. Couette flow
        o 2.1. velocity evolution
    o 3. Poissuille flow
        o 3.1. velocity evolution
    o 4. Flow in box with sliding lid
        o 4.1. stable simulation for long times
    o 5. Parallelization
        o 5.1. scaling plot
    
    Paper structure
    o 1. Overall structure
    o 2. Clarity of language
    o 3. Complete bibliography
    4. Clarity of figures
    o 5. Mathematical precision
    o 6. Reproducibility

    potential outline
    1. Introduction
    2. Methods
    3. Implementation
    4. Results
    5. Conclusion
\end{comment}

\chapter{Introduction}
\vspace{-8mm}
Large-scale physics experiments often require large budgets
and it is hard to perform experiments with several different parameters.
For this reason, many research has been performed to simulate real-world 
phenomenon.
One example is fluid flow and
fluid flow simulations allow us to deeply understand
how car body shapes relate to the aerodynamic drag
and how to optimize car designs through the simulations with
various designs rather than making real cars~\cite{padagannavar2016automotive}.
% Computational study of flow around a simplified car body

Such simulations require a scheme to simulate the physical states
at each time step
and the lattice Boltzmann method (LBM) ~\cite{timm2016lattice}
is one of the well-known
schemes for the fluid flow simulation method.
LBM approximates the physical states of a myriad of microscopic particles,
i.e. usually obtained by solving the Navier-Stokes equation,
by mesoscale physical states at each lattice grid.
The physical states or {\bf moments} are iteratively simulated based on
the Maxwell velocity distribution function~\cite{huang1963statistical} and
the fluid flow at each time step is derived from the moments.
The major advantages of LBM are the followings:
\begin{itemize}
  \item {\bf Simple implementation}: The governing equations of each moment
  are simple and the collision handling only considers the adjacent lattices. 
  \item {\bf Parallelization}: The LBM scales well with respect to
  the amount of parallel compuattional resources due to
  the local dynamics nature~\cite{raabe2004overview}
\end{itemize}
For those reasons, the LBM is one of the most successful methods and
we would like to introduce the LBM in this paper.
The paper structure is as follows:
\begin{enumerate}
  \item {\bf Lattice Boltzmann method (LBM) }: Show the theoretical aspects and
  how to discretize the equations
  \item {\bf Implementation}:
  Provide pseudocodes and how to efficiently compute
  the LBM
  \item {\bf Numerical results}~\footnote{
  The code is available at:
    https://github.com/nabenabe0928/high-performance-computing-fluid-dynamics-with-python
  }: Provide how we can validate the implementations
  and show how effective the parallel computation is
\end{enumerate}
All the codes follow {\tt pep8 style}~\footnote{https://www.python.org/dev/peps/pep-0008/} and 
are tested using
{\tt unittest}~\footnote{https://docs.python.org/3/library/unittest.html}.
Furthermore, {\bf the step-by-step reproduction instruction is available}
in {\tt README.md} on this repository.


\chapter{Lattice Boltzmann method}
\vspace{-8mm}
In this chapter, we describe how the equations used in LBM
are derived.
More specifically, we explain
{\bf Boltzmann transport equation (BTE)}\cite{mcnamara1988use}, i.e.
the basic equations of the kinetic theory of gases and
how to handle the boundary conditions.

\section{The Boltzmann transport equation (BTE)}
The BTE formulates the time evolution of the 
particle probability density $f(\xv, \uv, t)$ given
the velocity $\uv$ and the position $\xv$ of particles.
The BTE relaxes the particle distribution to
the Maxwell velocity distribution
function\cite{huang1963statistical} and the approximation of the relaxation of
$f$ towards $f^{\rm eq}$ is described as follows\cite{bhatnagar1954model}:
\begin{equation}
  \begin{aligned}
    \od{f(\xv, \uv, t)}{t} &= 
    - \frac{
      f(\xv, \uv, t) - \feq(\uv; \rho(\xv, t), \uv(\xv, t), T(\xv, t))
      }{\tau} \\
    \end{aligned}
    \label{analytical-eq}
  \end{equation}
where $f^{\rm eq}$ is statistical equilibrium,
$T(\xv, t)$ is the temperature at $\xv$
of time step $t$ and
$\tau$ is a characteristic time.
The characteristic time determines how quickly
the fluid converges towards the equilibrium.
The higher $\tau$ yields the slower 
convergence towards the equilibrium.
Eq~(\ref{analytical-eq}) is used for the update 
of the particle probability density function.
Furthermore, this particle probability density function
$f(\xv, \uv, t)$ is used for computing
the physical states of the fluid,
such as density and velocity.
The momentum updates are performed via\cite{caroli1984non}:
\begin{equation}
  \begin{aligned}
    \rho(\xv, t) &= \int f(\xv, \uv, t) d\uv 
  \end{aligned}
  \label{density-analytical-update}
\end{equation}
\begin{equation}
\begin{aligned}
  \uv(\xv, t) &= \frac{1}{\rho(\xv, t)} \int \uv f(\xv, \uv, t)  d\uv
\end{aligned}
\label{velocity-analytical-update}
\end{equation}
The underlying equations allow to simulate
fluid flow as seen in the latter parts of this paper.

\section{Time-step update of the BTE}
The aforementioned BTE is formulated in the 
continuous domain; therefore,
we need to discretize sptially and 
temporally to make the computation 
feasible by simulations.
In this paper, we focus on the discretization
in the two dimensional space.
The discretization for the space and time
is performed so that the equality condition of 
the following inequality
(Courant-Friedrichs-Lewy condition) holds\cite{peyretcomputational, sterling1996stability}:
\begin{equation}
\begin{aligned}
  \cv_i \dt \leq || \Delta \xv_i ||
\end{aligned}
\end{equation}
where $\dt$ is the time step size 
and $\Delta \xv_i$ is the distance between 
the closest grid in the direction
of $\cv_i$ that is defined:
\begin{equation}
\begin{aligned}
  \cv = \begin{bmatrix}
    0 & 1 & 0 & -1 & 0 & 1 & -1 & -1 & 1 \\
    0 & 0 & 1 & 0 & -1 & 1 & 1 & -1 & -1 \\
  \end{bmatrix}^\top
\end{aligned}
\label{d2q9-velocity}
\end{equation}
Note that this specific discretization in two-dimensional
space with nine direction shown in 
Figure~\ref{fig:d2q9} is called D2Q9.
In this setting, 
we first discretize
the particle probability density function
in nine directions by subscripting 
as $f_i(\xv, t)$.
Then Eq~(\ref{density-analytical-update}), (\ref{velocity-analytical-update})
become the follwoings:
\begin{equation}
  \begin{aligned}
    \rho(\xv, t) &= \sum_i f_i(\xv, t) \\
  \end{aligned}
  \label{discretized-density}
\end{equation}
\begin{equation}
\begin{aligned}
  \uv(\xv, t) &= 
  \frac{1}{\rho(\xv, t)} \sum_i \cv_i f_i(\xv) \\
\end{aligned}
\label{discretized-velocity}
\end{equation}
Note that we regard the density as
unit molecular mass in Eq~(\ref{discretized-density}).
Additionally, the equilibrium in Eq~(\ref{analytical-eq}) is computed as:
\begin{equation}
\begin{aligned}
  \underbrace{f_i(\xv + \cv_i\dt , t + \dt) - f_i(\xv, t)}_{
    \text{streaming}
  } &= 
  \underbrace{- \omega 
  \biggl[
    f_i(\xv, t) -
    \feq_i(\xv, t)
  \biggl]}_{
    \text{collision}
  }
\end{aligned}
\label{discretized-streaming}
\end{equation}
where $\omega = \dt / \tau$ is the relaxation parameter.
The equilibrium is computed as\cite{zhao2002non}:
\begin{equation}
\begin{aligned}
  \feq_i(\xv, t) &=
  w_i \rho(\xv, t) \biggl[
    1 + 3 \cv_i \cdot \uv(\xv, t) +
    \frac{9}{2}(\cv_i \cdot \uv(\xv, t))^2
    -\frac{3}{2} || \uv(\xv, t) ||^2
  \biggr] \\
\end{aligned}
\label{discretized-eq}
\end{equation}
where the index $i$ corresponds to Figure~\ref{fig:d2q9}
and the weights are 
\begin{eqnarray}
  w_i = \left\{
  \begin{array}{cl}
     \frac{4}{9}&~(i = 0) \\
     \frac{1}{9}&~(i=1,2,3,4) \\
     \frac{1}{36}&~(i=5,6,7,8)
  \end{array}
  \right.
\end{eqnarray}
In the streaming step, the grid receives 
the particle flow $f_i(\xv + \cv_i \dt, \cdot)$
from its nine adjacent grids.
In the collision step,
we relax the probability density function 
towards the equilibrium $f_i^{\rm eq}$
by considering the effects of the particle collision.

\begin{figure}[h!]
  \begin{center}
   \includegraphics[width=10cm]{logos/Gitter_LBM.png}
   \caption{
      (a) The discretization on the velocity space according to D2Q9.
      (b) The uniform two-dimensional grids for
      the discretization in the physical space.
   }
  \label{fig:d2q9}
  \end{center}
\end{figure}

\section{Boundary handling}
In this section, we briefly discuss how we handle
the particles that bump into boundaries.
Note that the boundary handling is performed
after the streaming step that is discussed in the previous section
and we usually use the direction that is opposite to
the direction $i$ for the bounce back.
For this reason, we will denote
$f^\star_i$ as the $i$-th direction
particle probability density function
after the streaming step
and $i^\star$ as the direction opposite,
i.e. {\bf reflected direction}, to $i$.
Those directions follow D2Q9 illustrated
in Figure~\ref{fig:d2q9}.
Additionally, there are the following
two ways to
implement the boundary conditions\cite{liu2014lattice}:
\begin{enumerate}
  \item {\bf Dry nodes}:
  The boundaries are located on the link between nodes
  \item {\bf Wet nodes}:
  The boundaries are located on the lattice nodes
\end{enumerate}
Since the boundary handling will be tedious when
the boundaries are placed on the lattice nodes,
and this is the case for wet nodes,
we use {\bf dry nodes} for the implementation.

\subsection{Bounce-back from objects}
The most basic boundary condition is 
{\bf rigid wall} or the {\bf bounce-back boundary condition}.
In this condition, we apply the process without
slip condition at the boundary.
The equation at the boundary is computed as\cite{succi2018lattice}:
\begin{equation}
\begin{aligned}
  f_i(\xv_b, t + \dt) = f_{i^\star}^\star(\xv_b, t)
\end{aligned}
\label{discretized-rigid-wall}
\end{equation}
When the {\bf boundary moves} with the velocity of
$\Uv_w$, the variation in the momentum of particles
must be taken into consideration and the equation is
modified as follows\cite{succi2018lattice}:
\begin{equation}
  \begin{aligned}
    f_i(\xv_b, t + \dt) = f_{i^\star}^\star (\xv_b, t) - 
    2 w_i \rho_w \frac{
      \cv_i \cdot \Uv_w
    }{c_s^2}
  \end{aligned}
  \label{discretized-moving-wall}
\end{equation}
where $c_s$ is the speed of sound and 
$\rho_w$ is the density at the wall.
The computation of $\rho_w$ is usually performed by
either of the followings\cite{zou1997pressure, khajepor2019study}:
\begin{enumerate}
  \item Take the average density $\bar{\rho}$ of the simulated field
  \item Extrapolate $\rho_w$ using 
  the particle probability density function in the physical domain
\end{enumerate}
For the simplicity, we {\bf take the first solution}.

\subsection{Periodic boundary conditions (PBC)}
In this section, we assume that we have
boundary at $x = 0~(\text{inlet}), X - 1~(\text{outlet})$
where $X$ is the number of the lattice grid in the $x$-axis.
The most basic PBC assumes that
the flow from outlet comes in from inlet as follows\cite{succi2018lattice}: 
\begin{equation}
\begin{aligned}
  f((0, y), t) = f((X - 1, y), t)
\end{aligned}
\end{equation}
This condition is implicitly implemented during the streaming operation.
Another PBC handles
the pressure variation $\Delta p$ between inlet and outlet.
Since the density $\rho$ is computed using the pressure $p$
as $\rho = \frac{p}{c_s^2}$ where $c_s$ is the speed of sound,
the density at the inlet $\rho_{\rm in}$ and
that at the outlet $\rho_{\rm out}$ can be computed
accordingly given the constant pressure $p_{\rm out}$
at the outlet.
Then the particle probability density functions at 
the inlet and the outlet are computed as follows\cite{succi2018lattice}:
\begin{equation}
\begin{aligned}
  f_i^\star(0, y, t) &=
  f_i^{\rm eq}(\rho_{\rm in}, \uv(X - 1, y, t))
  + (f_i^\star(X - 1, y, t) - f_i^{\rm eq}(X - 1, y, t))\\
  f_i^\star(X - 1, y, t) &=
  f_i^{\rm eq}(\rho_{\rm out}, \uv(0, y, t))
  + (f_i^\star(0, y, t) - f_i^{\rm eq}(0, y, t))\\
\end{aligned}
\label{discretized-pbc-pressure}
\end{equation}


\chapter{Implementation}
\vspace{-5mm}
In this chapter, we describe how the LBM is implemented in Python
and how to compute the LBM in parallel.
All the implementation is assuming that
the physical domain is discretized by D2Q9
and the horizontal axis is $x$ and 
the vertical axis is $y$, respectively.
Since the most important
Note that entire codes are based on
Numpy\footnote{Numpy: https://numpy.org/}
and mpi4py\footnote{mpi4py: https://mpi4py.readthedocs.io/en/stable/}.
Throughout the chapter, {\tt numpy} is imported as {\tt np}.

\section{Main routine}
Algorithm~\ref{alg:lattice-boltzmann-method-algorithm}
shows the pseudocode of the main processing in the LBM.
Recall that $f(\xv, t)$.shape = $(X, Y, 9)$,
$\rho(\xv, 0)$.shape = $(X, Y)$ and $\uv(\cdot, 0)$.shape = $(X, Y, 2)$, 
Note that the order of each step might vary depending on literature\cite{}.
First, we provide the initial values for the density and the velocity.
Then, we compute the probability function and equillibrium and
apply the collision step.
The equillibrium implementation is shown in Algorithm~\ref{alg:equillibrium-algorithm}.
After applying equillibrium, we perform
streaming operation shown in Algorithm~\ref{alg:streaming-algorithm}
and slide each quantity to the adjacent cells.
Finally, we apply the boundary handling at each boundary cell as 
described in Algorithm~\ref{alg:boundary-conditions-algorithm}
and update the density and the velocity as in Eq~\ref{}.
Note that we use as much slicing as possible in 
Algorithms to speed up the runtime as much as possible.

The Algorithm~\ref{alg:streaming-algorithm} uses
the {\tt np.roll} operation that enables
to handle periodic boundary conditions automatically.
This function rolls the array in the following manner:
\begin{equation}
\begin{aligned}
  \text{np.roll}(f[x][y][i], \text{shift}=\cv_i, \text{axis}=(0, 1)) =
  f[nx][ny][i] \\
  \text{where }
  nx = (x + \cv_i[0]) \% X,
  ny = (y + \cv_i[1]) \% Y
\end{aligned}
\end{equation}
where $i$ is the direction index in D2Q9 and $\cv_i$ is the vector
that specifies the $i$-th direction in D2Q9.
In the Algorithm~\ref{alg:boundary-conditions-algorithm},
we use boolean matrices {\tt in\_boundary} and {\tt out\_boundary}
that have the shape of $(X, Y, 9)$ so that 
we can refer to only elements that have bounce back or collision
with the boundary.
Additionally, we compute $\rho_w$ by the average density in Eq~\ref{}.
Note that the domain is extended with virtual nodes 
at both edges of the periodic boundary in the PBC with pressure variation
so that the implementation is more straightforward. 


\begin{algorithm}[tb]
  \caption{The main routine of the lattice Boltzmann method}
  \label{alg:lattice-boltzmann-method-algorithm}
  \begin{algorithmic}[1]
    \Statex{The grid size: $X, Y$,
    Relaxation factor : $\omega$,
    Initial velocity: $\uv_0$,
    Initial density: $\rho_0$
    } \Comment{Inputs}
    \Statex{Boundary conditions}
    \Function{lattice boltzmann method}{}
    \State{$\rho(\xv, 0) = \rho_0, \uv(\xv, 0) = \uv_0$ for all $\xv \in [0, X) \times [0, Y)$}
    \For{$t= 0, 1, \dots$}
    \State{$\feq(\cdot, t),f^\star(\cdot, t)$ = equillibrium($f(\cdot, t), \rho(\cdot, t), \uv(\cdot, t)$)}
    \Comment{Eq~\ref{}}
    \State{$f^\star$ = $f + \omega (\feq - f)$}
    \Comment{Eq~\ref{}}
    \State{$f^\star(\cdot + \cv \dt, t)$ =streaming($f(\cdot, t), \feq(\cdot, t)$)}
    \Comment{Eq~\ref{}}
    \State{$f(\cdot, t + 1)$ = boundary\_handling($f^\star(\cdot, t),\feq(\cdot, t),\text{**kwargs}$)}
    \Comment{Eq~\ref{}}
    \State{$\rho(\cdot, t + 1), \uv(\cdot, t + 1)$=moments\_update($f(\cdot, t + 1)$)}
    \Comment{Eq~\ref{}}
    \EndFor
    \EndFunction
  \end{algorithmic}
\end{algorithm}

\begin{algorithm}[tb]
  \caption{Equillibrium}
  \label{alg:equillibrium-algorithm}
  \begin{algorithmic}[1]
    \Statex{w = $ \text{np.array([}
    \frac{4}{9}, \frac{1}{9}, \frac{1}{9}, 
    \frac{1}{9}, \frac{1}{9}, \frac{1}{36}, 
    \frac{1}{36}, \frac{1}{36}, \frac{1}{36}
    \text{])}$, c = $\cv$ in Eq~\ref{}}
    \Function{equillibrium}{$\rho$ = $\rho(\cdot, t)$, u = $\uv(\cdot, t)$}
    \Comment{$\uv$.shape = $(X, Y, 2)$, $\rho$.shape = $(X, Y)$}
    \State{u\_norm2 = (u ** 2).sum(axis=-1)[..., None]}
    \State{u\_at\_c = u @ c.T}
    \Comment{u\_at\_c.shape = $(X, Y, 9)$}
    \State{w\_tmp, $\rho$\_{tmp} = w[None, None, ...], $\rho$[..., None]}
    \Comment{Adapt the shapes to u\_at\_c}
    \State{$\feq$ = w\_tmp * $\rho$\_tmp * (1 + 3 * u\_at\_c + 4.5 * (u\_at\_c) ** 2)-1.5 * u\_norm2}
    \State{{\bf return} $\feq$}
    \EndFunction
  \end{algorithmic}
\end{algorithm}

\begin{algorithm}[tb]
  \caption{Streaming operation}
  \label{alg:streaming-algorithm}
  \begin{algorithmic}[1]
    \Statex{c = $\cv$ in Eq~\ref{}}
    \Function{streaming}{$f^\star$ = $f^\star(\cdot, t)$}
    \State{$f^{\rm post}$ = np.zeros\_like($f^\star$)}
    \For{$i= 0, 1, \dots, 8$}
    \State{$f^{\rm post}[..., i]$=np.roll($f^\star$[..., i], shift=c[i], axis=(0, 1))}
    \Comment{Slide $f^\star$ one step to c[i]}
    \EndFor
    \State{{\bf return} $f^{\rm post}$}
    \EndFunction
  \end{algorithmic}
\end{algorithm}

\begin{algorithm}[tb]
  \caption{Boundary conditions}
  \label{alg:boundary-conditions-algorithm}
  \begin{algorithmic}[1]
    \Statex{
      Boolean matrix that represents
      where we have the bounce back: in\_boundary
    }
    \Statex{
      Boolean matrix that represents
      where we have the collision: out\_boundary
    }
    \Statex{
      The indices in D2Q9 s.t. the flow comes in
      given boundaries: in\_indices
    }
    \Statex{
      The indices in D2Q9 s.t. the flow goes out
      given boundaries: out\_indices
    }
    \Function{boundary handlling}{$f^\star$ = $f^\star(\cdot, t)$,
      $\feq$ = $\feq(\cdot, t)$}
    \If{Rigid wall}
    \State{$f$[in\_boundary] = $f^\star$[out\_boundary]}
    \EndIf
    \If{Moving wall}
    \State{coef = np.zeros\_like(out\_boundary)}
    \State{{\bf for} out\_idx, ci, wi in zip(out\_indices, c, w) {\bf do}}
    \Indent
    \State{coef[:, :, out\_idx] = 2 * wi * (ci @ $\uv_w$)} / c\_s ** 2
    \EndIndent
    \State{$f$[in\_boundary] = $f^\star$[out\_boundary]-$\rho_w$[out\_boundary] * coef[out\_boundary]}
    \EndIf
    \If{PBC with pressure variation}
    \Comment{fluid flows from $x = 0$ to $X - 1$}
    \State{$\feq_{\rm in}, \feq_{\rm out}$ = equilibrium($\rho_{\rm in}$, $\uv$[-2]), equilibrium($\rho_{\rm out}$, $\uv$[1])}
    \State{$f^{\rm post}$[0][:,out\_indices]=$\feq_{\rm in}$[:,out\_indices]+($f$[-2][:,out\_indices]-$\feq$[-2][:,out\_indices])}
    \State{$f^{\rm post}$[-1][:, in\_indices]=$\feq_{\rm out}$[:, in\_indices]+($f$[1][:, in\_indices] - $\feq$[1][:, in\_indices])}
    \EndIf
    \State{{\bf return} $f$}
    \EndFunction
  \end{algorithmic}
\end{algorithm}

\section{Parallel computation by MPI}\label{section-mpi}
In order to process the LBM in parallel,
we employ the spatial domain decomposition and
the messaging passing interface (MPI)
so that we can compute collision step of the LBM
in parallel.
Note that the collision step does not require
any communication between processes.
When we are provided the number of processes of $P$,
we first factorize $P$ such that $P = P_0 \times P_1$
where $P_0, P_1 \in \mathbb{Z}^{+}$
and $P_0, P_1 = \text{arg} \min_{P_0, P_1}(|| P_1 - P_0 ||)$.
Then, we divide the $x$-axis into $P_0$ intervals and
$y$-axis into $P_1$ intervals where
any pairs of intervals $I_{i}, I_{j}$ in a direction satisfies
$-1 \leq || I_{i} || - ||I_{j}|| \leq 1$.
Note that $||I||$ is the size of the interval $I$.
This split of the domain achieves the most balanced distribution of
the computation.
For the streaming step, we need to consider particles
moving from one process to another.
We implement it using so-called {\bf ghost cells}
around the actual computational domain.
Figure~\ref{} shows the conceptual visualization of
how each process communicate and ghost cells work.
Since each process requires the four edges of adjacent
processes, the commnications are required four times
for each process.
The Algorithm~\ref{alg:mpi-algorithm}
shows the implementation using mpi4py.
{\tt grid\_manager} is our self-developed module
that allows to get useful information related to
process location, the adjacent relation and so on.
{\tt Sendrecv} function is used for the commnication and
each process receives an array from {\tt dest} that is sent
by {\tt neighbor} and sends an array {\tt sendbuf} to
{\tt neighbor}.
Note that {\tt buf} is the abbreviation of buffer and
it is literally used for the buffer to communicate data.


\begin{algorithm}[tb]
  \caption{The communication of
  the particle probability density function}
  \label{alg:mpi-algorithm}
  \begin{algorithmic}[1]
    \Statex{Process and lattice grids management: grid\_manager}
    \Function{communication}{}
    \State{{\bf for} dir in grid\_manager.neighbor\_directions {\bf do}}
    \Comment{Iterate over the D2Q9 index}
    \Indent
    \State{dx, dy = $\cv_i$}
    \State{sendidx = grid\_manager.step\_to\_idx(dx, dy, send=True)}
    \State{recvidx = grid\_manager.step\_to\_idx(dx, dy, send=False)}
    \State{neighbor = grid\_manager.get\_neighbor\_rank(dir)}
    \If{dx == 0} \Comment{send to top and bottom}
    \State{sendbuf = $f$[:, sendidx, ...].copy()}
    \State{grid\_manager.rank\_grid.Sendrecv(sendbuf=sendbuf, dest=neighbor,}
    \State{\hspace{56mm} recvbuf=recvbuf, source=neighbor)}
    \State{$f$[:, recvidx, ...] = recvbuf}
    \ElsIf{dy == 0} \Comment{send to left and right}
    \State{sendbuf = $f$[sendidx, ...].copy()}
    \State{grid\_manager.rank\_grid.Sendrecv(sendbuf=sendbuf, dest=neighbor,}
    \State{\hspace{56mm} recvbuf=recvbuf, source=neighbor)}
    \State{$f$[recvidx, ...] = recvbuf}
    \Else
    \State{sendbuf = $f$[sendidx[0], sendidx[1] ...].copy()}
    \State{grid\_manager.rank\_grid.Sendrecv(sendbuf=sendbuf, dest=neighbor,}
    \State{\hspace{56mm} recvbuf=recvbuf, source=neighbor)}
    \State{$f$[recvidx[0], recvidx[1], ...] = recvbuf}
    \EndIf
    \EndIndent
    \State{{\bf return} $f$}
    \EndFunction
  \end{algorithmic}
\end{algorithm}

\chapter{Numerical results}
\vspace{-8mm}
In the previous chapter, we discuss the implementation details
and how we implement and apply LBM to various settings.
In this chapter, we show the visualizations and numerical results
obtained from the series experiments.

\section{Validation experiments}
In the physics simulation, it is always important
to validate whether the implementation is correct.
Therefore, we first show how to validate the implementation
using several examples.

\subsection{Shear wave decay}
The shear wave decay represents the time evolution of a
velocity perturbation in the flow.
Since the viscosity decays the velocity of the flow,
the velocity converges to zero in the end.
When we set the following sinusoidal perturbation in the velocity
as the initial condition:
\begin{equation}
  \begin{aligned}
    \uv(\xv, t = 0) =
    \begin{bmatrix}
      u_x(y, t = 0) \\
      0 \\
    \end{bmatrix}
    =  
    \begin{bmatrix}
      \epsilon \sin \frac{2\pi y}{Y} \\
        0 \\
      \end{bmatrix}
  \end{aligned}
  \label{shear-vel-init}
\end{equation}
Then the analytical solution for the time evolution of 
the velocity is calculated as follows~\cite{fei2018three}:
\begin{equation}
  \begin{aligned}
    u_x(y, t) &= 
    \epsilon \exp\biggl(
      -\nu \biggl(
        \frac{2\pi}{Y}
      \biggr)^2 t\biggr) \sin \frac{2\pi y}{Y}
  \end{aligned}
  \label{sinusoidal-vel-analytical-solution}
\end{equation}
Note that this result is obtained using Navier-Stokes equations for incompressible fluid
and the assumptions that the pressure term $\nabla p$ and
the convection term $(\uv \cdot \nabla) \uv$ are negligible 
compared to the viscosity term $\nu \nabla^2 \uv$.
In Figure~\ref{fig:sinusoidal-velocity}, we show the plot of both
simulated results and the analytical solutions of sinusoidal velocity.
Note that the initial condition follows Eq~(\ref{shear-vel-init}). 
As seen in the figure, the simulated results and the analytical solutions
{\bf perfectly fit} and thus we could validate our implementation of
rigid wall and moments updates.
Figure~\ref{fig:sinusoidal-density} shows the density distribution
over time.
This simulation uses the sinusoidal density in $x$-direction:
\begin{equation}
\begin{aligned}
  \rho(\xv, 0) = \rho_0 + \epsilon \sin \frac{2 \pi x}{X}
\end{aligned}
\end{equation}

\begin{figure}[H]
  \begin{center}
    \subfloat[$t = 0$]{
      \includegraphics[width=0.23\textwidth]{../log/sinusoidal_velocity/fig/vel000000.pdf}
    }
    \subfloat[$t = 200$]{
      \includegraphics[width=0.23\textwidth]{../log/sinusoidal_velocity/fig/vel000200.pdf}
    }
    \subfloat[$t = 400$]{
      \includegraphics[width=0.23\textwidth]{../log/sinusoidal_velocity/fig/vel000400.pdf}
    }
    \subfloat[$t = 600$]{
      \includegraphics[width=0.23\textwidth]{../log/sinusoidal_velocity/fig/vel000600.pdf}
    }\\
    \vspace{-3mm}
    \subfloat[$t = 800$]{
      \includegraphics[width=0.23\textwidth]{../log/sinusoidal_velocity/fig/vel000800.pdf}
    }
    \subfloat[$t = 1000$]{
      \includegraphics[width=0.23\textwidth]{../log/sinusoidal_velocity/fig/vel001000.pdf}
    }
    \subfloat[$t = 1200$]{
      \includegraphics[width=0.23\textwidth]{../log/sinusoidal_velocity/fig/vel001200.pdf}
    }
    \subfloat[$t = 1400$]{
      \includegraphics[width=0.23\textwidth]{../log/sinusoidal_velocity/fig/vel001400.pdf}
    }\\
    \vspace{-3mm}
    \subfloat[$t = 1600$]{
      \includegraphics[width=0.23\textwidth]{../log/sinusoidal_velocity/fig/vel001600.pdf}
    }
    \subfloat[$t = 1800$]{
      \includegraphics[width=0.23\textwidth]{../log/sinusoidal_velocity/fig/vel001800.pdf}
    }
    \subfloat[$t = 2000$]{
      \includegraphics[width=0.23\textwidth]{../log/sinusoidal_velocity/fig/vel002000.pdf}
    }
    \subfloat[$t = 2200$]{
      \includegraphics[width=0.23\textwidth]{../log/sinusoidal_velocity/fig/vel002200.pdf}
    }\\
    \caption{The velocity evolution for the sinusoidal velocity at
      the $x = 25$ in the lattice grid size of $(50, 50)$.
      The $x$-axis shows the location in the $y$ direction
      and the $y$-axis shows the magnitude of velocity at 
      the corresponding location.
      The coefficients $\epsilon$ and the initial density $\rho_0$ are 
      set to $0.01$ and $1.0$ respectively.
      The relaxation term $\omega$ are set to $1.0$.
      \label{fig:sinusoidal-velocity}}
  \end{center}
\end{figure}

Additionally, we obtain the following equation regarding the viscosity by
transforming Eq~(\ref{sinusoidal-vel-analytical-solution}):
\begin{equation}
  \begin{aligned}
    u_x(y, t) &= \epsilon
    \Biggl(
    -\nu
    \biggl(
      \frac{2\pi}{Y}
      \biggr)^2 t
    \Biggr) 
    \sin  \frac{2\pi y}{Y} \exp
      \Longleftrightarrow 
    \frac{u_x(y, t)}{
      \epsilon
      \sin  \frac{2\pi y}{Y}
    }  =  \exp
    \Biggl(
    -\nu
    \biggl(
      \frac{2\pi}{Y}
      \biggr)^2 t
    \Biggr) \\
    -\nu
    \biggl(
      \frac{2\pi}{Y}
      \biggr)^2 t
      &= 
    \log \frac{u_x(y, t)}{
      \epsilon
      \sin  \frac{2\pi y}{Y}
    }~(\text{Take log of both sides}) \\
    \nu
      &=
      - \frac{1}{t}
      \biggl(
        \frac{Y}{2\pi}
        \biggr)^2 
    \log \frac{u_x(y, t)}{
      \epsilon
      \sin  \frac{2\pi y}{Y}
    } \\
  \end{aligned}
  \label{viscosity-analytical}
\end{equation}
Note that we assume that $\epsilon\sin \frac{2\pi y}{Y} \neq 0$
and the assumptions for Eq~(\ref{sinusoidal-vel-analytical-solution}) hold.
We perform the experiments to validate Eq~(\ref{viscosity-analytical}) using 
the exact experiment settings for Figure~\ref{fig:sinusoidal-velocity} and Figure~\ref{fig:sinusoidal-density}
except the relaxation term $\omega$.
Note that the viscosity is computed as $\nu = \frac{1}{3} (\frac{1}{\omega} - \frac{1}{2})$.
The results are shown in Figure~\ref{fig:omega-vs-visc}.
Based on the results, smaller and larger $\omega$ lead to numerical instability.
Otherwise, the simulated results and analytical solution fit perfectly.
Therefore, we need to avoid using $\omega$ closer to $0$ or $2$ for more accurate results.

\begin{figure}[H]
  \begin{center}
    \subfloat[$t = 0$]{
      \includegraphics[width=0.23\textwidth]{../log/sinusoidal_density/fig/density000000.pdf}
    }
    \subfloat[$t = 200$]{
      \includegraphics[width=0.23\textwidth]{../log/sinusoidal_density/fig/density000200.pdf}
    }
    \subfloat[$t = 400$]{
      \includegraphics[width=0.23\textwidth]{../log/sinusoidal_density/fig/density000400.pdf}
    }
    \subfloat[$t = 600$]{
      \includegraphics[width=0.23\textwidth]{../log/sinusoidal_density/fig/density000600.pdf}
    }\\
    \vspace{-3mm}
    \subfloat[$t = 800$]{
      \includegraphics[width=0.23\textwidth]{../log/sinusoidal_density/fig/density000800.pdf}
    }
    \subfloat[$t = 1000$]{
      \includegraphics[width=0.23\textwidth]{../log/sinusoidal_density/fig/density001000.pdf}
    }
    \subfloat[$t = 1200$]{
      \includegraphics[width=0.23\textwidth]{../log/sinusoidal_density/fig/density001200.pdf}
    }
    \subfloat[$t = 1400$]{
      \includegraphics[width=0.23\textwidth]{../log/sinusoidal_density/fig/density001400.pdf}
    }\\
    \vspace{-3mm}
    \subfloat[$t = 1600$]{
      \includegraphics[width=0.23\textwidth]{../log/sinusoidal_density/fig/density001600.pdf}
    }
    \subfloat[$t = 1800$]{
      \includegraphics[width=0.23\textwidth]{../log/sinusoidal_density/fig/density001800.pdf}
    }
    \subfloat[$t = 2000$]{
      \includegraphics[width=0.23\textwidth]{../log/sinusoidal_density/fig/density002000.pdf}
    }
    \subfloat[$t = 2200$]{
      \includegraphics[width=0.23\textwidth]{../log/sinusoidal_density/fig/density002200.pdf}
    }\\
    \caption{The density evolution for the sinusoidal density at
      the $y = 25$ in the lattice grid size of $(50, 50)$.
      The $x$-axis shows the location in the $x$ direction
      and the $y$-axis shows the magnitude of density at 
      the corresponding location.
      The coefficients $\epsilon$ and $\rho_0$ are 
      set to $0.01$ and $1.0$ respectively.
      The relaxation term $\omega$ are set to $1.0$.
      \label{fig:sinusoidal-density}}
  \end{center}
\end{figure}

\begin{figure}[H]
  \begin{center}
    \subfloat[Sinusoidal density]{
      \includegraphics[width=0.48\textwidth]{../log/sinusoidal_density/fig/omega_vs_visc.pdf}
    }
    \subfloat[Sinusoidal velocity]{
      \includegraphics[width=0.48\textwidth]{../log/sinusoidal_velocity/fig/omega_vs_visc.pdf}
    }
    \caption{The simulated viscosity value 
    over various relaxation values $\omega$.
    The analytical solution uses $\nu = \frac{1}{3}(\frac{1}{\omega} - \frac{1}{2})$
    and the simulated result uses Eq~(\ref{viscosity-analytical}).
    For the simulated results, we take the results at $t = 3000$
    and compute the average of $\nu$ at $X = 25$ over the $y$-axis for sinusoidal velocity
    and at $Y = 25$ over the $x$-axis for sinusoidal density.
    Note that (a) uses the same parameters as in Figure~\ref{fig:sinusoidal-density}
    and (a) uses the same parameters as in Figure~\ref{fig:sinusoidal-velocity}.
    \label{fig:omega-vs-visc}}
  \end{center}
\end{figure}

\begin{figure}[H]
  \centering
  \includegraphics[width=0.98\textwidth]{imgs/couette_and_poiseuille.pdf}
  \caption{The conceptual visualizations of the Couette flow (Left) and
  Poiseuille flow (Right).}
  \label{couette-and-poiseuille-conceptual}
\end{figure}


\subsection{Couette flow}
The Couette flow is the flow between two walls as shown in
Figure~\ref{couette-and-poiseuille-conceptual}:
One is fixed and the other moves horizontally with the velocity of $U_w$.
The flow is caused by the viscous drag force acting on the fluid.
Since the Couette flow also has an analytical solution,
we can validate the implementation of the moving wall.
The analytical solution for Figure~\ref{couette-and-poiseuille-conceptual} is given by~\cite{nagy2019graphical}:
\begin{equation}
\begin{aligned}
  u_x(y) =\frac{Y - y}{Y}U_w
\end{aligned}
\end{equation}
where $Y$ is the distance between the two walls
and $u_x(y)$ is the horizontal velocity of the flow
at the location of $y$. 
In the experiment, we apply the bounce-back boundary condition
at the moving wall and the rigid wall
and the periodic boundary conditions at the inlet and outlet.
The results are shown in Figure~\ref{fig:couette-velocity-evolution}.
As shown in the figures, the flow velocity iteratively approaches
the analytical solution and it perfectly fits in the end
and the velocity stops growing as seen at $t=1600, 2000$.
From this experiment, the moving wall can be validated.

\begin{figure}[H]
  \vspace{-1mm}
  \centering
  \includegraphics[width=0.58\textwidth]{../log/couette_flow/fig/couette_flow_joint.pdf}
  \vspace{-5mm}
  \caption{The velocity evolution at
  the $x = 25$ in the lattice grid size of $(50, 50)$.
  The wall velocity $U_w$ at the bottom and the relaxation term $\omega$ are set
  to $50$ and $0.3$ respectively.
  The initial density is $\rho(\xv) = 1.0$ and the initial velocity is $\uv(\xv) = (0, 0)$. 
  \label{fig:couette-velocity-evolution}}
\end{figure}

\subsection{Poiseuille flow}
The Poiseuille flow is the flow between two non-moving walls as shown in Figure~\ref{couette-and-poiseuille-conceptual}.
The flow is caused by a constant pressure difference $\od{p}{x}$
in the horizontal direction of the two walls.
The Poiseuille flow also has the analytical solution
and we can validate the implementation of the periodic boundary conditions
with pressure variation.
The analytical solution for Figure~\ref{couette-and-poiseuille-conceptual} is given by~\cite{mendiburu2009analytical}:
\begin{equation}
\begin{aligned}
  u_x(y) = - \frac{1}{2\rho \nu} \od{p}{x} y (Y - y)
\end{aligned}
\end{equation}
In the experiment, we apply the bounce-back boundary condition
at the moving wall and the rigid wall
and the periodic boundary condition with pressure variation at the inlet and outlet.
Figure~\ref{fig:poiseuille-velocity-evolution} presents the results
and the simulated results approaches the analytical solutions as in the Couette flow.
In the end, it fits completely
and the velocity stops growing as seen at $t=4000, 5000$.

\begin{figure}[H]
  \centering
  \includegraphics[width=0.58\textwidth]{../log/poiseuille_flow/fig/poiseuille_flow_joint.pdf}
  \caption{The velocity evolution at
  the $x = 25$ in the lattice grid size of $(50, 50)$.
  The relaxation term $\omega$ are is
  to $0.7$.
  The density factor at the inlet and the density factor
  at the outlet are set to $0.3$ and $0.301$ respectively.
  The initial density is $\rho(\xv) = 1.0$ and the initial velocity is $\uv(\xv) = (0, 0)$.
  \label{fig:poiseuille-velocity-evolution}}
\end{figure}

\section{Lid-driven cavity}
Finally, we handle a concrete example.
In this paper, the lid-driven cavity shown in Figure~\ref{lid-driven-cavity-conceptual} are simulated.
The lid-driven cavity simulates the flow inside a box with
three rigid walls and one moving wall, i.e. a lid.
In this simulation, it is known that the turbulence is caused 
when the following Reynolds number is larger than 1000~\cite{chiang1998effect}:
\begin{equation}
\begin{aligned}
  \text{Re} = \frac{LU}{\nu}
\end{aligned}
\end{equation}
where $L$ is the characteristic length parameter
of the body and $U$ is the stream flow velocity.
One key property of the Reynolds number is that two flow system
is dynamically similar if the Reynolds number and the geometry are similar~\cite{kundu2008fluid}.
Therefore, we present the results with various 
viscosity $\nu$ and the wall velocity $U = U_w$ that 
satisfy the Reynolds number of $1000$ under $L = X = Y = 300$
in Figure~\ref{fig:sliding-lid-velocity-evolution}.
As the viscosity and velocity becomes larger,
the convergence becomes quicker.
On the other hand, all the settings converge to the similar flow in the end
as indicated in the key property of the Reynolds number.

\begin{figure}[t]
  \centering
  \includegraphics[width=0.48\textwidth]{imgs/lid-driven-cavity.pdf}
  \caption{The conceptual visualizations of the lid-driven cavity.}
  \label{lid-driven-cavity-conceptual}
\end{figure}

\begin{figure}[t]
  \begin{center}
    \subfloat[$t = 5000$]{
      \includegraphics[width=0.20\textwidth]{../log/sliding_lid_W0.10_visc0.03_size300/fig/vel_flow005000.pdf}
    }
    \subfloat[$t = 10000$]{
      \includegraphics[width=0.20\textwidth]{../log/sliding_lid_W0.10_visc0.03_size300/fig/vel_flow010000.pdf}
    }
    \subfloat[$t = 50000$]{
      \includegraphics[width=0.20\textwidth]{../log/sliding_lid_W0.10_visc0.03_size300/fig/vel_flow050000.pdf}
    }
    \subfloat[$t = 100000$]{
      \includegraphics[width=0.20\textwidth]{../log/sliding_lid_W0.10_visc0.03_size300/fig/vel_flow095000.pdf}
    }\\
    \vspace{-3mm}
    \subfloat[$t = 5000$]{
      \includegraphics[width=0.20\textwidth]{../log/sliding_lid_W0.20_visc0.06_size300/fig/vel_flow005000.pdf}
    }
    \subfloat[$t = 10000$]{
      \includegraphics[width=0.20\textwidth]{../log/sliding_lid_W0.20_visc0.06_size300/fig/vel_flow010000.pdf}
    }
    \subfloat[$t = 50000$]{
      \includegraphics[width=0.20\textwidth]{../log/sliding_lid_W0.20_visc0.06_size300/fig/vel_flow050000.pdf}
    }
    \subfloat[$t = 100000$]{
      \includegraphics[width=0.20\textwidth]{../log/sliding_lid_W0.20_visc0.06_size300/fig/vel_flow095000.pdf}
    }\\
    \vspace{-3mm}
    \subfloat[$t = 5000$]{
      \includegraphics[width=0.20\textwidth]{../log/sliding_lid_W0.30_visc0.09_size300/fig/vel_flow005000.pdf}
    }
    \subfloat[$t = 10000$]{
      \includegraphics[width=0.20\textwidth]{../log/sliding_lid_W0.30_visc0.09_size300/fig/vel_flow010000.pdf}
    }
    \subfloat[$t = 50000$]{
      \includegraphics[width=0.20\textwidth]{../log/sliding_lid_W0.30_visc0.09_size300/fig/vel_flow050000.pdf}
    }
    \subfloat[$t = 100000$]{
      \includegraphics[width=0.20\textwidth]{../log/sliding_lid_W0.30_visc0.09_size300/fig/vel_flow095000.pdf}
    }\\
    \vspace{-3mm}
    \subfloat[$t = 5000$]{
      \includegraphics[width=0.20\textwidth]{../log/sliding_lid_W0.40_visc0.12_size300/fig/vel_flow005000.pdf}
    }
    \subfloat[$t = 10000$]{
      \includegraphics[width=0.20\textwidth]{../log/sliding_lid_W0.40_visc0.12_size300/fig/vel_flow010000.pdf}
    }
    \subfloat[$t = 50000$]{
      \includegraphics[width=0.20\textwidth]{../log/sliding_lid_W0.40_visc0.12_size300/fig/vel_flow050000.pdf}
    }
    \subfloat[$t = 100000$]{
      \includegraphics[width=0.20\textwidth]{../log/sliding_lid_W0.40_visc0.12_size300/fig/vel_flow095000.pdf}
    }\\
    \caption{The stream plots of sliding lid
    with the lattice grid size of $(300, 300)$.
    Figure~(a) -- (d) are the results of 
    the viscosity $\nu = 0.03$ and the wall velocity $0.1$.
    Figure~(e) -- (h) are the results of 
    the viscosity $\nu = 0.06$ and the wall velocity $0.2$.
    Figure~(i) -- (l) are the results of 
    the viscosity $\nu = 0.09$ and the wall velocity $0.3$.
    Figure~(m) -- (p) are the results of 
    the viscosity $\nu = 0.12$ and the wall velocity $0.4$.
    Note that each setting is chosen to satisfy
    the Reynolds number 1000.
    The initial density is $\rho(\xv) = 1.0$ and the initial velocity is $\uv(\xv) = (0, 0)$.
      \label{fig:sliding-lid-velocity-evolution}}
  \end{center}
\end{figure}

This experiment requires long time to complete.
For example, it takes 1 hour to finish one simulation using
intel core i7--10700 and 32GB RAM.
Recall that the advantage of LBM is to allow us to compute the simulation in
parallel easily.
For this reason, we test the scalability of this simulation using
various number of processes.
Note that all the experiments related to the scaling test
are performed on {\bf BWUniCluster}
\footnote{https://wiki.bwhpc.de/e/Category:BwUniCluster\_2.0}.
The implementation follows Section~\ref{section-mpi}.
Figure~\ref{fig:sliding-lid-scaling} shows the plot of
MLUPS, a.k.a. million lattice updates per second, and
the number of processes.
Ideally, the MLUPS grows linearly with respect to the number of processes.
However, it does not happen because of the latency of the communication
and the waiting for the synchronization as described in Amdahl's law~\cite{amdahl1967validity}.
Such slow down is also seen in Figure~\ref{fig:sliding-lid-scaling}.
\begin{figure}[b]
  \centering
  \includegraphics[width=0.78\textwidth]{../log/sliding_lid_W0.10_visc0.03_size300/fig/scaling_test.pdf}
  \vspace{-3mm}
  \caption{The scaling test of the sliding lid simulation.
  The grid size is $300 \times 300$.
  The number of processes are $1 ,4 ,9 ,16 ,25 ,36 ,100 ,144 ,225 ,400$ respectively.
  Note that both axes are log-scale.
    The viscosity and the wall velocity are set to $\nu = 0.03$ and $0.1$.
    The initial density is $\rho(\xv) = 1.0$ and the initial velocity is $\uv(\xv) = (0, 0)$.
  }
  \label{fig:sliding-lid-scaling}
\end{figure}



\chapter{Conclusions}
\vspace{-8mm}
In this paper,
we delineate the theoretical aspects of the LBM
and the implementations of the LBM.
Chapter 1 describes the motivation behind the numerical integrations
and 
we explain that the advantages of the LBM
are the simple implementation
and scalability with respect to the computational
resources.

Chapter 2 explains the theoretical aspects of LBM
and how those equations are plugged into two-dimensional
computational simulations.
The governing equation of the particle movement is
the BTE and the BTE relaxes the particle distribution
to the Maxwell velocity distribution.
Thereafter, we show the approximation and how we obtain each moment, i.e.
physical states such as density or velocity
from the particle distribution.
Then the discretization of each equation and
the boundary handlings are presented.
In the descriptions, we also add the
reasons behind some tricks used in the implementations.

Chapter 3 shows the algorithms of each component.
Especially, we focus on the explanation of how
the simulation should be implemented using numpy
which is effective to speed up {\tt Python} implementations.
Additionally, the MPI usage and the domain division method
are described.
The domain decomposition is performed so that the 
load balance is optimized.
Note that each algorithm in the implementations
is tested using {\tt unittest} and abstracted as much as possible
so that each component can be reused and we can reduce
the bugs over the whole implementation.
Additionally, we provide the running scripts and {\tt requirements.txt}
to reproduce the experimental settings.

Chapter 4 presents the validations of each component
using the comparison between the analytical solutions
and visualizes how the LBM works in the lid-driven cavity
example.
For the validations, we use the shear wave decay, 
the Couette flow and the Poissuille flow.
The results show that the simulated results
coincide with the analytical solutions except for
the cases where the relaxation term $\omega$ is 
close to either $0$ or $2$.
After the validations, we perform the lid-driven cavity simulation
and the result exhibits similar dynamics when we have a constant Reynolds number
and more noisy turbulence as the Reynolds number becomes larger.
Furthermore, we test the scalability of the LBM in the lid-driven cavity simulation.
The experiments show speedup in all the settings compared to
the serial implementations.
On the other hand, the smaller domains have
less efficiency and they even slow down as the number of processes increases.
This observation corresponds to the intuition from Amdahl's law due to
the latency of the communication and the waiting for the synchronization.
Recall that the parallel implementation is tested by the direct
comparison of the velocity field with the identical settings
as the serial implementation. 


\bibliographystyle{unsrt}
\bibliography{biblio}

\end{document}
