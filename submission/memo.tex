\documentclass[dvipdfmx]{article}
\usepackage[utf8]{inputenc}
\usepackage{amsmath,amssymb,bm,bbm}
\usepackage{graphicx}
\usepackage{ascmac}
\usepackage{fancyhdr}
\usepackage{graphicx}
\usepackage{algorithm, algpseudocode}
\usepackage{algpseudocode}
\usepackage{tikz}
\usepackage{ulem}
\usepackage{booktabs}
\usepackage{multirow}
\usepackage[caption=false]{subfig}
\usepackage{comment}
\usepackage{listings}
\usetikzlibrary{chains}
\usetikzlibrary{calc}
\usepackage{amsmath,tikz}
\usepackage{lastpage}
\usepackage{tcolorbox}
\usepackage{cancel}
\tcbuselibrary{breakable, skins, theorems}
\newtheorem{theorem1}{Theorem}
\newtheorem{theorem2}{Definition}
\newtheorem{theorem3}{Assumption}
\def\qed{\hfill $\Box$}

% remove the end from algorithm
\algdef{SE}[SUBALG]{Indent}{EndIndent}{}{\algorithmicend\ }%
\algtext*{Indent}
\algtext*{EndIndent}
\algtext*{EndFor}
\algtext*{EndWhile}
\algtext*{EndIf}
\algtext*{EndProcedure}
\algtext*{EndFunction}
\newcommand{\Break}{\textbf{break}}
\newcommand{\Continue}{\textbf{continue}}

\newcommand{\xv}{\boldsymbol{x}}
\newcommand{\vv}{\boldsymbol{v}}
\newcommand{\uv}{\boldsymbol{u}}
\newcommand{\wv}{\boldsymbol{w}}
\newcommand{\pd}[2]{\frac{\partial #1}{\partial #2}}
\newcommand{\od}[2]{\frac{d #1}{d #2}}
\newcommand{\pdtwo}[2]{\frac{\partial^2 #1}{\partial #2^2}}
\newcommand{\biggNorm}[1]{\biggl|\biggl| #1 \biggr|\biggr|}

\newcommand{\cv}{\boldsymbol{c}}
\newcommand{\feq}{f^{\rm eq}}
\newcommand{\dt}{\Delta t}

\setlength{\textwidth}{156mm}
\setlength{\textheight}{220mm}
\setlength{\oddsidemargin}{5mm}
\setlength{\voffset}{-15mm}
\setlength{\headsep}{10mm}
\pagestyle{fancy}
\rhead{\thepage/\pageref{LastPage}}
\cfoot{ }

\lhead[The Subject name / Shuhei Watanabe]{The subject name / Shuhei Watanabe}

\title{\vspace{-15mm} The Subject name}
\author{Shuhei Watanabe}
\date{\today}

\begin{document}
\maketitle
\thispagestyle{fancy}

\begin{comment}
                 init_pdf: Optional[np.ndarray] = None,
                 init_density: Optional[np.ndarray] = None,
                 init_vel
\end{comment}

\begin{algorithm}[tb]
  \caption{The main routine of the lattice Boltzmann method}
  \label{alg:lattice-boltzmann-method-algorithm}
  \begin{algorithmic}[1]
    \Statex{The grid size: $X, Y$,
    Relaxation factor : $\omega$,
    Initial velocity: $\uv_0$,
    Initial density: $\rho_0$
    } \Comment{Inputs}
    \Statex{Boundary conditions}
    \Function{lattice boltzmann method}{}
    \State{$\rho(\xv, 0) = \rho_0, \uv(\xv, 0) = \uv_0$ for all
    $\xv \in [0, X) \times [0, Y)$}
    \For{$t= 0, 1, \dots$}
    \State{$\feq(\cdot, t),
    f^\star(\cdot, t)
    $ = equillibrium($f(\cdot, t), \rho(\cdot, t), \uv(\cdot, t)$)}
    \Comment{Eq~\ref{}}
    \State{$f^\star$ = $f + \omega (\feq - f)$}
    \Comment{Eq~\ref{}}
    \State{$f^\star(\cdot + \cv \dt, t)$ =
    streaming($f(\cdot, t), \feq(\cdot, t)$)}
    \Comment{Eq~\ref{}}
    \State{$f(\cdot, t + 1)$ = 
    boundary\_handling($
    f^\star(\cdot, t),
    \feq(\cdot, t),
    \text{**kwargs}
    $)}
    \Comment{Eq~\ref{}}
    \State{$\rho(\cdot, t + 1), \uv(\cdot, t + 1)$ =
    moments\_update($f(\cdot, t + 1)$)}
    \Comment{Eq~\ref{}}
    \EndFor
    \EndFunction
  \end{algorithmic}
\end{algorithm}

\begin{algorithm}[tb]
  \caption{Equillibrium}
  \label{alg:equillibrium-algorithm}
  \begin{algorithmic}[1]
    \Statex{w = $ \text{np.array([}
    \frac{4}{9}, \frac{1}{9}, \frac{1}{9}, 
    \frac{1}{9}, \frac{1}{9}, \frac{1}{36}, 
    \frac{1}{36}, \frac{1}{36}, \frac{1}{36}
    \text{])}$, c = $\cv$ in Eq~\ref{}}
    \Function{equillibrium}{$\rho$ = $\rho(\cdot, t)$, u = $\uv(\cdot, t)$}
    \Comment{$\uv$.shape = $(X, Y, 2)$, $\rho$.shape = $(X, Y)$}
    \State{u\_norm2 = (u ** 2).sum(axis=-1)[..., None]}
    \State{u\_at\_c = u @ c.T}
    \Comment{u\_at\_c.shape = $(X, Y, 9)$}
    \State{w\_tmp, $\rho$\_{tmp} = w[None, None, ...], $\rho$[..., None]}
    \Comment{Adapt the shapes to u\_at\_c}
    \State{$\feq$ = w\_tmp * $\rho$\_tmp * 
    (1 + 3 * u\_at\_c + 4.5 * (u\_at\_c) ** 2)
    -1.5 * u\_norm2
    }
    \State{{\bf return} $\feq$}
    \EndFunction
  \end{algorithmic}
\end{algorithm}

\begin{algorithm}[tb]
  \caption{Streaming operation}
  \label{alg:streaming-algorithm}
  \begin{algorithmic}[1]
    \Statex{c = $\cv$ in Eq~\ref{}}
    \Function{streaming}{$f^\star$ = $f^\star(\cdot, t)$}
    \State{$f^{\rm post}$ = np.zeros\_like($f^\star$)}
    \For{$i= 0, 1, \dots, 8$}
    \State{$f^{\rm post}[..., i]$ =
    np.roll($f^\star$[..., i], shift=c[i], axis=(0, 1))}
    \Comment{Slide $f^\star$ one step to c[i]}
    \EndFor
    \State{{\bf return} $f^{\rm post}$}
    \EndFunction
  \end{algorithmic}
\end{algorithm}

\begin{algorithm}[tb]
  \caption{Boundary conditions}
  \label{alg:boundary-conditions-algorithm}
  \begin{algorithmic}[1]
    \Statex{
      Boolean matrix that represents
      where we have the bounce back: in\_boundary
    }
    \Statex{
      Boolean matrix that represents
      where we have the collision: out\_boundary
    }
    \Statex{
      The indices in D2Q9 s.t. the flow comes in
      given boundaries: in\_indices
    }
    \Statex{
      The indices in D2Q9 s.t. the flow goes out
      given boundaries: out\_indices
    }
    \Function{boundary handlling}{$f^\star$ = $f^\star(\cdot, t)$,
      $\feq$ = $\feq(\cdot, t)$}
    \If{Rigid wall}
    \State{$f$[in\_boundary] = $f^\star$[out\_boundary]}
    \EndIf
    \If{Moving wall}
    \State{coef = np.zeros\_like(out\_boundary)}
    \State{{\bf for} out\_idx, ci, wi in zip(out\_indices, c, w) {\bf do}}
    \Indent
    \State{coef[:, :, out\_idx] = 2 * wi * (ci @ $\uv_w$)} / c\_s ** 2
    \EndIndent
    \State{$f$[in\_boundary] = $f^\star$[out\_boundary]
    - $\rho_w$[out\_boundary] * coef[out\_boundary]}
    \EndIf
    \If{PBC with pressure variation}
    \Comment{fluid flows from $x = 0$ to $X - 1$}
    \State{$\feq_{\rm in}, \feq_{\rm out}$ = equilibrium($\rho_{\rm in}$, $\uv$[-2]), equilibrium($\rho_{\rm out}$, $\uv$[1])}
    \State{
      $f^{\rm post}$[0][:, out\_indices] = 
      $\feq_{\rm in}$[:, out\_indices] + 
      ($f$[-2][:, out\_indices] - $\feq$[-2][:, out\_indices])
    }
    \State{
      $f^{\rm post}$[-1][:, in\_indices] = 
      $\feq_{\rm out}$[:, in\_indices] + 
      ($f$[1][:, in\_indices] - $\feq$[1][:, in\_indices])
    }
    \EndIf
    \State{{\bf return} $f$}
    \EndFunction
  \end{algorithmic}
\end{algorithm}

\begin{algorithm}[tb]
  \caption{The communication of
  the particle probability density function}
  \label{alg:mpi-algorithm}
  \begin{algorithmic}[1]
    \Statex{Process and lattice grids management: grid\_manager}
    \Function{communication}{}
    \State{{\bf for} dir in grid\_manager.neighbor\_directions {\bf do}}
    \Comment{Iterate over the D2Q9 index}
    \Indent
    \State{dx, dy = $\cv_i$}
    \State{sendidx = grid\_manager.step\_to\_idx(dx, dy, send=True)}
    \State{recvidx = grid\_manager.step\_to\_idx(dx, dy, send=False)}
    \State{neighbor = grid\_manager.get\_neighbor\_rank(dir)}
    \If{dx == 0} \Comment{send to top and bottom}
    \State{sendbuf = $f$[:, sendidx, ...].copy()}
    \State{grid\_manager.rank\_grid.Sendrecv(sendbuf=sendbuf, dest=neighbor,}
    \State{\hspace{50mm} recvbuf=recvbuf, source=neighbor)}
    \State{$f$[:, recvidx, ...] = recvbuf}
    \ElsIf{dy == 0} \Comment{send to left and right}
    \State{sendbuf = $f$[sendidx, ...].copy()}
    \State{grid\_manager.rank\_grid.Sendrecv(sendbuf=sendbuf, dest=neighbor,}
    \State{\hspace{50mm} recvbuf=recvbuf, source=neighbor)}
    \State{$f$[recvidx, ...] = recvbuf}
    \Else
    \State{sendbuf = $f$[sendidx[0], sendidx[1] ...].copy()}
    \State{grid\_manager.rank\_grid.Sendrecv(sendbuf=sendbuf, dest=neighbor,}
    \State{\hspace{50mm} recvbuf=recvbuf, source=neighbor)}
    \State{$f$[recvidx[0], recvidx[1], ...] = recvbuf}
    \EndIf
    \EndIndent
    \State{{\bf return} $f$}
    \EndFunction
  \end{algorithmic}
\end{algorithm}


\bibliographystyle{splncs04}
\bibliography{ref}
\end{document}