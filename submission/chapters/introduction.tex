\chapter{Introduction}
\vspace{-8mm}
Large-scale physics experiments often require large budgets
and it is hard to perform experiments with several different parameters.
For this reason, many research has been performed to simulate real-world 
phenomenon.
One of the phenomenon is the fluid flow.
For example, fluid flow simulations allow us to deeply understand
how the car body shape relates to the aerodynamic drag
and to optimize the car design through the simulations with
various designs rather than making real cars~\cite{padagannavar2016automotive}.
% Computational study of flow around a simplified car body

Such simulations require scheme to simulate the physical states
at each time step
and the lattice Boltzmann method (LBM) ~\cite{timm2016lattice}
is one of the well-known
schemes for the fluid flow simulation method.
LBM approximates the physical states of a myriad of microscopic particles,
i.e. usually obtained by solving Navier-Stokes equation,
by mesoscale physical states at each lattice grid.
The physical states or {\bf moments} are iteratively simulated based on
the Maxwell velocity distribution function~\cite{huang1963statistical} and
the fluid flow at each time step is derived from the moments.

The major advantages of LBM are the followings:
\begin{itemize}
  \item {\bf Simple implementation}: The governing equations of each moment
  is simple and the collision handling only considers the adjacent lattices. 
  \item {\bf Parallelization}: The parallelization scales well due to
  the local dynamics nature of LBM~\cite{raabe2004overview}
\end{itemize}
For those reasons, LBM is one of the most successful methods and
we would like to introduce LBM in detail in this paper.
The paper structure is as follows:
\begin{enumerate}
  \item {\bf Lattice boltzmann method}; Show the governing equations and 
  provide pseudocodes to promote the understandings
  \item {\bf Numerical results}\footnote{
  The code is available at:
    https://github.com/nabenabe0928/high-performance-computing-fluid-dynamics-with-python
  }: Provide how we can validate the implementations,
  and how effective the parallel computation is
\end{enumerate}
All the codes follow the pep8 style
\footnote{https://www.python.org/dev/peps/pep-0008/} and 
are tested using
unittest\footnote{https://docs.python.org/3/library/unittest.html}.
Furthermore, {\bf the step-by-step reproduction instruction is available}
in {\tt README.md} on this repository.
