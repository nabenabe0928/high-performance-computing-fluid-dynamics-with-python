\chapter{Numerical results}
\vspace{-5mm}
In the previous chapter, we discuss the implementation detail
and how we implement and apply LBM to various settings.
In this chapter, we show the visualizations and numerical results
obtained from the series experiments.

\section{Validation experiments}
In the physics simulation, it is always important
to validate whether the implementation is correct
\cite{}.
Therefore, we first show how to validate the implementation
using several examples.

\subsection{Shear wave decay}
The shear wave decay represents the time evolution of a
velocity perturbation in the flow.
Since the viscosity decays the velocity of the flow,
the velocity converges to zero in the end.
In order to obtain an analytical solution,
we employ the following Navier-Stokes equations for incompressible fluid
\cite{}:
\begin{equation}
  \begin{aligned}
    \text{\bf
      Navier-Stokes equation
    }:&~ \pd{\uv}{t} = - (\uv \cdot \nabla)\uv
    - \frac{1}{\rho} \nabla p
    + \nu \nabla^2 \uv\\
    \text{\bf
      Conservation of mass
    }: &~
    \nabla \cdot \uv = 0
  \end{aligned}
\end{equation}
where $p$ is the pressure and $\nu$ is 
the kinematic viscosity.
Note that we need to assume the following
constraints to obtain the analytical solutions:
\begin{equation}
\begin{aligned}
 || (\uv \cdot \nabla)\uv || \ll \biggNorm{\pd{\uv}{t}},
  || \nabla p || \ll \biggNorm{\pd{\uv}{t}}
\end{aligned}
\end{equation}
Then we can derive the following equation:
\begin{equation}
\begin{aligned}
  \pd{\uv}{t} = \nu \nabla^2 \uv
\end{aligned}
\label{relaxed-ns-eq}
\end{equation}
Let $X, Y$ be the width and height of the box
where we simulate the fluid flow
and we set the following sinusoidal perturbation
in the velocity as the initial condition:
\begin{equation}
\begin{aligned}
  \uv(\xv, t = 0) =
  \begin{bmatrix}
    u_x(y, t = 0) \\
    0 \\
  \end{bmatrix}
  =  
  \begin{bmatrix}
      a_0 \sin \frac{2\pi y}{Y} \\
      0 \\
    \end{bmatrix}
\end{aligned}
\label{shear-vel-init}
\end{equation}
Using Eq~(\ref{relaxed-ns-eq}) and Eq~(\ref{shear-vel-init}),
the following analytical solution is derived:
\begin{equation}
\begin{aligned}
  \pd{u_x(y, t)}{t} &= \nu \pdtwo{u_x(y, t)}{y} \\
  \pd{a(t)}{t}v(y) &= \nu a(t) v^{(2)}(y) ~(u_x(y, t) \triangleq a(t)v(y) ) \\
  \pd{a(t)}{t} \cancel{\sin \frac{2\pi y}{Y}}
  &= -\nu a(t) \biggl(
    \frac{2\pi y}{Y}
  \biggr)^2
  \cancel{\sin \frac{2\pi y}{Y}} ~\biggl(\because \frac{d^2 \sin bx}{dx^2} = -b^2 \sin bx
  \biggr) \\
  \therefore a(t) &= a_0 \exp\biggl(
    -\nu \biggl(
      \frac{2\pi y}{Y}
    \biggr)^2 t
  \biggr) \\
  \implies u_x(y, t) &= 
  a_0 \exp\biggl(
    -\nu \biggl(
      \frac{2\pi y}{Y}
    \biggr)^2 t\biggr) \sin \frac{2\pi y}{Y}
\end{aligned}
\end{equation}


\begin{figure}[tb]
  \begin{center}
    \subfloat[$t = 0$]{
      \includegraphics[width=0.23\textwidth]{../log/sinusoidal_density/fig/density000000.pdf}
    }
    \subfloat[$t = 200$]{
      \includegraphics[width=0.23\textwidth]{../log/sinusoidal_density/fig/density000200.pdf}
    }
    \subfloat[$t = 400$]{
      \includegraphics[width=0.23\textwidth]{../log/sinusoidal_density/fig/density000400.pdf}
    }
    \subfloat[$t = 600$]{
      \includegraphics[width=0.23\textwidth]{../log/sinusoidal_density/fig/density000600.pdf}
    }\\
    \subfloat[$t = 800$]{
      \includegraphics[width=0.23\textwidth]{../log/sinusoidal_density/fig/density000800.pdf}
    }
    \subfloat[$t = 1000$]{
      \includegraphics[width=0.23\textwidth]{../log/sinusoidal_density/fig/density001000.pdf}
    }
    \subfloat[$t = 1200$]{
      \includegraphics[width=0.23\textwidth]{../log/sinusoidal_density/fig/density001200.pdf}
    }
    \subfloat[$t = 1400$]{
      \includegraphics[width=0.23\textwidth]{../log/sinusoidal_density/fig/density001400.pdf}
    }\\
    \subfloat[$t = 1600$]{
      \includegraphics[width=0.23\textwidth]{../log/sinusoidal_density/fig/density001600.pdf}
    }
    \subfloat[$t = 1800$]{
      \includegraphics[width=0.23\textwidth]{../log/sinusoidal_density/fig/density001800.pdf}
    }
    \subfloat[$t = 2000$]{
      \includegraphics[width=0.23\textwidth]{../log/sinusoidal_density/fig/density002000.pdf}
    }
    \subfloat[$t = 2200$]{
      \includegraphics[width=0.23\textwidth]{../log/sinusoidal_density/fig/density002200.pdf}
    }\\
    \caption{The density evolution for the Sinusoidal density at
      the $Y = 25$ in the lattice grid size of $(50, 50)$.
      The coefficients $\epsilon$ and $\rho_0$ are 
      set to $0.08$ and $0.5$ respectively.
      The relaxation term $\omega$ are set to $1.0$.
      \label{fig:sinusoidal-density}}
  \end{center}
\end{figure}


\begin{figure}[tb]
  \begin{center}
    \subfloat[$t = 0$]{
      \includegraphics[width=0.23\textwidth]{../log/sinusoidal_velocity/fig/vel000000.pdf}
    }
    \subfloat[$t = 200$]{
      \includegraphics[width=0.23\textwidth]{../log/sinusoidal_velocity/fig/vel000200.pdf}
    }
    \subfloat[$t = 400$]{
      \includegraphics[width=0.23\textwidth]{../log/sinusoidal_velocity/fig/vel000400.pdf}
    }
    \subfloat[$t = 600$]{
      \includegraphics[width=0.23\textwidth]{../log/sinusoidal_velocity/fig/vel000600.pdf}
    }\\
    \subfloat[$t = 800$]{
      \includegraphics[width=0.23\textwidth]{../log/sinusoidal_velocity/fig/vel000800.pdf}
    }
    \subfloat[$t = 1000$]{
      \includegraphics[width=0.23\textwidth]{../log/sinusoidal_velocity/fig/vel001000.pdf}
    }
    \subfloat[$t = 1200$]{
      \includegraphics[width=0.23\textwidth]{../log/sinusoidal_velocity/fig/vel001200.pdf}
    }
    \subfloat[$t = 1400$]{
      \includegraphics[width=0.23\textwidth]{../log/sinusoidal_velocity/fig/vel001400.pdf}
    }\\
    \subfloat[$t = 1600$]{
      \includegraphics[width=0.23\textwidth]{../log/sinusoidal_velocity/fig/vel001600.pdf}
    }
    \subfloat[$t = 1800$]{
      \includegraphics[width=0.23\textwidth]{../log/sinusoidal_velocity/fig/vel001800.pdf}
    }
    \subfloat[$t = 2000$]{
      \includegraphics[width=0.23\textwidth]{../log/sinusoidal_velocity/fig/vel002000.pdf}
    }
    \subfloat[$t = 2200$]{
      \includegraphics[width=0.23\textwidth]{../log/sinusoidal_velocity/fig/vel002200.pdf}
    }\\
    \caption{The velocity evolution for the Sinusoidal velocity at
      the $x = 25$ in the lattice grid size of $(50, 50)$.
      The coefficients $\epsilon$ and $\rho_0$ are 
      set to $0.08$ and $0.5$ respectively.
      The relaxation term $\omega$ are set to $1.0$.
      \label{fig:sinusoidal-velocity}}
  \end{center}
\end{figure}

\begin{figure}[tb]
  \begin{center}
    \subfloat[Sinusoidal density]{
      \includegraphics[width=0.48\textwidth]{../log/sinusoidal_density/fig/omega_vs_visc.pdf}
    }
    \subfloat[Sinusoidal velocity]{
      \includegraphics[width=0.48\textwidth]{../log/sinusoidal_velocity/fig/omega_vs_visc.pdf}
    }
    \caption{The simulated viscosity value 
    over various relaxation values $\omega$.\label{fig:omega-vs-visc}}
  \end{center}
\end{figure}

\begin{equation}
  \begin{aligned}
    \vv_x(y, t) & = \epsilon \exp
    \Biggl(
    -\nu
    \biggl(
      \frac{2\pi}{Y}
      \biggr)^2 t
    \Biggr) 
    \sin 
      \frac{2\pi y}{Y} \\
    \frac{\vv_x(y, t)}{
      \epsilon
      \sin  \frac{2\pi y}{Y}
    } & =  \exp
    \Biggl(
    -\nu
    \biggl(
      \frac{2\pi}{Y}
      \biggr)^2 t
    \Biggr) \\
    -\nu
    \biggl(
      \frac{2\pi}{Y}
      \biggr)^2 t
      &= 
    \log \frac{\vv_x(y, t)}{
      \epsilon
      \sin  \frac{2\pi y}{Y}
    } \\
    \nu
      &=
      - \frac{1}{t}
      \biggl(
        \frac{Y}{2\pi}
        \biggr)^2 
    \log \frac{\vv_x(y, t)}{
      \epsilon
      \sin  \frac{2\pi y}{Y}
    } \\
  \end{aligned}
\end{equation}

\subsection{Couette flow}

\subsubsection{Problem settings}

\subsubsection{Visualizations}

\begin{figure}[tb]
  \begin{center}
    \subfloat[$t = 0$]{
      \includegraphics[width=0.44\textwidth]{../log/couette_flow/fig/couette_flow000000.pdf}
    }
    \subfloat[$t = 200$]{
      \includegraphics[width=0.44\textwidth]{../log/couette_flow/fig/couette_flow000200.pdf}
    }\\
    \subfloat[$t = 400$]{
      \includegraphics[width=0.44\textwidth]{../log/couette_flow/fig/couette_flow000400.pdf}
    }
    \subfloat[$t = 600$]{
      \includegraphics[width=0.44\textwidth]{../log/couette_flow/fig/couette_flow000600.pdf}
    }\\
    \subfloat[$t = 800$]{
      \includegraphics[width=0.44\textwidth]{../log/couette_flow/fig/couette_flow000800.pdf}
    }
    \subfloat[$t = 1000$]{
      \includegraphics[width=0.44\textwidth]{../log/couette_flow/fig/couette_flow001000.pdf}
    }\\
    \subfloat[$t = 1200$]{
      \includegraphics[width=0.44\textwidth]{../log/couette_flow/fig/couette_flow001200.pdf}
    }
    \subfloat[$t = 1400$]{
      \includegraphics[width=0.44\textwidth]{../log/couette_flow/fig/couette_flow001400.pdf}
    }
    \caption{The velocity evolution at
      the $x = 25$ in the lattice grid size of $(50, 50)$.
      The wall velocity and the relaxation term $\omega$ are set
      to $50$ and $0.3$ respectively.
      \label{fig:couette-velocity-evolution}}
  \end{center}
\end{figure}


\subsection{Poiseuille flow}

\subsubsection{Problem settings}

\subsubsection{Visualizations}

\begin{figure}[tb]
  \begin{center}
    \subfloat[$t = 0$]{
      \includegraphics[width=0.31\textwidth]{../log/poiseuille_flow/fig/poiseuille_flow000000.pdf}
    }
    \subfloat[$t = 500$]{
      \includegraphics[width=0.31\textwidth]{../log/poiseuille_flow/fig/poiseuille_flow000500.pdf}
    }
    \subfloat[$t = 1000$]{
      \includegraphics[width=0.31\textwidth]{../log/poiseuille_flow/fig/poiseuille_flow001000.pdf}
    }\\
    \subfloat[$t = 1500$]{
      \includegraphics[width=0.31\textwidth]{../log/poiseuille_flow/fig/poiseuille_flow001500.pdf}
    }
    \subfloat[$t = 2000$]{
      \includegraphics[width=0.31\textwidth]{../log/poiseuille_flow/fig/poiseuille_flow002000.pdf}
    }
    \subfloat[$t = 2500$]{
      \includegraphics[width=0.31\textwidth]{../log/poiseuille_flow/fig/poiseuille_flow002500.pdf}
    }\\
    \subfloat[$t = 3000$]{
      \includegraphics[width=0.31\textwidth]{../log/poiseuille_flow/fig/poiseuille_flow003000.pdf}
    }
    \subfloat[$t = 3500$]{
      \includegraphics[width=0.31\textwidth]{../log/poiseuille_flow/fig/poiseuille_flow003500.pdf}
    }
    \subfloat[$t = 4000$]{
      \includegraphics[width=0.31\textwidth]{../log/poiseuille_flow/fig/poiseuille_flow004000.pdf}
    }
    \caption{The velocity evolution at
      the $x = 25$ in the lattice grid size of $(50, 50)$.
      The relaxation term $\omega$ are is
      to $0.7$.
      The density factor at the inlet and the density factor
      at the outlet are set to $0.3$ and $0.301$ respectively.
      \label{fig:poiseuille-velocity-evolution}}
  \end{center}
\end{figure}

\section{Concrete example (lid-driven cavity)}
\subsection{Problem settings}

\subsection{Visualizations}
\begin{figure}[tb]
  \begin{center}
    \subfloat[$t = 5000$]{
      \includegraphics[width=0.23\textwidth]{../log/sliding_lid_W0.10_visc0.03_size300/fig/vel_flow005000.pdf}
    }
    \subfloat[$t = 10000$]{
      \includegraphics[width=0.23\textwidth]{../log/sliding_lid_W0.10_visc0.03_size300/fig/vel_flow010000.pdf}
    }
    \subfloat[$t = 50000$]{
      \includegraphics[width=0.23\textwidth]{../log/sliding_lid_W0.10_visc0.03_size300/fig/vel_flow050000.pdf}
    }
    \subfloat[$t = 100000$]{
      \includegraphics[width=0.23\textwidth]{../log/sliding_lid_W0.10_visc0.03_size300/fig/vel_flow095000.pdf}
    }\\
    \subfloat[$t = 5000$]{
      \includegraphics[width=0.23\textwidth]{../log/sliding_lid_W0.20_visc0.06_size300/fig/vel_flow005000.pdf}
    }
    \subfloat[$t = 10000$]{
      \includegraphics[width=0.23\textwidth]{../log/sliding_lid_W0.20_visc0.06_size300/fig/vel_flow010000.pdf}
    }
    \subfloat[$t = 50000$]{
      \includegraphics[width=0.23\textwidth]{../log/sliding_lid_W0.20_visc0.06_size300/fig/vel_flow050000.pdf}
    }
    \subfloat[$t = 100000$]{
      \includegraphics[width=0.23\textwidth]{../log/sliding_lid_W0.20_visc0.06_size300/fig/vel_flow095000.pdf}
    }\\
    \subfloat[$t = 5000$]{
      \includegraphics[width=0.23\textwidth]{../log/sliding_lid_W0.30_visc0.09_size300/fig/vel_flow005000.pdf}
    }
    \subfloat[$t = 10000$]{
      \includegraphics[width=0.23\textwidth]{../log/sliding_lid_W0.30_visc0.09_size300/fig/vel_flow010000.pdf}
    }
    \subfloat[$t = 50000$]{
      \includegraphics[width=0.23\textwidth]{../log/sliding_lid_W0.30_visc0.09_size300/fig/vel_flow050000.pdf}
    }
    \subfloat[$t = 100000$]{
      \includegraphics[width=0.23\textwidth]{../log/sliding_lid_W0.30_visc0.09_size300/fig/vel_flow095000.pdf}
    }\\
    \subfloat[$t = 5000$]{
      \includegraphics[width=0.23\textwidth]{../log/sliding_lid_W0.40_visc0.12_size300/fig/vel_flow005000.pdf}
    }
    \subfloat[$t = 10000$]{
      \includegraphics[width=0.23\textwidth]{../log/sliding_lid_W0.40_visc0.12_size300/fig/vel_flow010000.pdf}
    }
    \subfloat[$t = 50000$]{
      \includegraphics[width=0.23\textwidth]{../log/sliding_lid_W0.40_visc0.12_size300/fig/vel_flow050000.pdf}
    }
    \subfloat[$t = 100000$]{
      \includegraphics[width=0.23\textwidth]{../log/sliding_lid_W0.40_visc0.12_size300/fig/vel_flow095000.pdf}
    }\\
    \caption{The stream plots of sliding lid
    with the lattice grid size of $(300, 300)$.
    Figure~(a) -- (d) are the results of 
    the viscosity $\nu = 0.03$ and the wall velocity $0.1$.
    Figure~(e) -- (h) are the results of 
    the viscosity $\nu = 0.06$ and the wall velocity $0.2$.
    Figure~(i) -- (l) are the results of 
    the viscosity $\nu = 0.09$ and the wall velocity $0.3$.
    Figure~(m) -- (p) are the results of 
    the viscosity $\nu = 0.12$ and the wall velocity $0.4$.
    Note that each setting is chosens to satisfy
    the Reynolds number 1000.
      \label{fig:sliding-lid-velocity-evolution}}
  \end{center}
\end{figure}

\subsection{Scaling test}
\begin{figure}
  \centering
  \includegraphics[width=0.68\textwidth]{../log/sliding_lid_W0.10_visc0.03_size300/fig/scaling_test.pdf}
  \caption{The scaling test of the sliding lid simulation.
  The grid size is $300 \times 300$.
  Note that both axes are log-scale.
  }
  \label{fig:sliding-lid-scaling}
\end{figure}

