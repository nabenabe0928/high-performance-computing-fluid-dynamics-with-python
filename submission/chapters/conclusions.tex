\chapter{Conclusions}
In this paper, we describe what the LBM is
and how we implement it.
Chapter 1 describes the motivation behind the numerical integrations
and mentions the advantages of the LBM, i.e.
simple implementation and scalability with respect to the computational
resources.

Chapter 2 explains the theoretical aspects of LBM
and how those equations are plugged into the computational simulations.
More specifically, the discretization of each equation and
the boundary handlings are presented.

Chapter 3 shows the algorithms of each component.
Especially, we focus on the descriptions how
the simulation should be implemented using numpy
which is effective to speed up {\tt Python} implementations.
Additionally, the MPI usage and the domain division method
are described. 
Note that each algorithm in our implementation
is tested using {\tt unittest} and abstracted as much as possible
so that each component can be reused and we can reduce
the bugs over the whole implementation.

Chapter 4 presents the validations of each component
using the comparison between the analytical solutions
and visualizes how the LBM works in the lid-driven cavity
example.
For the validations, we use the shear wave decay, 
the Couette flow and the Poissuille flow
and show the analytical solutions for each phenomenon.
After the validations, we show the lid-driven cavity simulation
exhibits similar dynamics when we have a constant Reynolds number.
Furthermore, we test the scalability of the LBM in the lid-driven cavity simulation.
The experiments show that
the 30 times speedup using 100 processes for $300 \times 300$ grids
and bad scalability in the smaller grids' settings.
This observation corresponds to the intuition from Amdahl's law.
