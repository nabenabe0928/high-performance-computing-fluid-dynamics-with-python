\chapter{Conclusions}
\vspace{-8mm}
In this paper,
we delineate the theoretical aspects of the LBM
and the implementations of the LBM.
Chapter 1 describes the motivation behind the numerical integrations
and 
we explain that the advantages of the LBM
are the simple implementation
and scalability with respect to the computational
resources.

Chapter 2 explains the theoretical aspects of LBM
and how those equations are plugged into two-dimensional
computational simulations.
The governing equation of the particle movement is
the BTE and the BTE relaxes the particle distribution
to the Maxwell velocity distribution.
Thereafter, we show the approximation and how we obtain each moment, i.e.
physical states such as density or velocity
from the particle distribution.
Then the discretization of each equation and
the boundary handlings are presented.
In the descriptions, we also add the
reasons behind some tricks used in the implementations.

Chapter 3 shows the algorithms of each component.
Especially, we focus on the explanation of how
the simulation should be implemented using numpy
which is effective to speed up {\tt Python} implementations.
Additionally, the MPI usage and the domain division method
are described.
The domain decomposition is performed so that the 
load balance is optimized.
Note that each algorithm in the implementations
is tested using {\tt unittest} and abstracted as much as possible
so that each component can be reused and we can reduce
the bugs over the whole implementation.
Additionally, we provide the running scripts and {\tt requirements.txt}
to reproduce the experimental settings.

Chapter 4 presents the validations of each component
using the comparison between the analytical solutions
and visualizes how the LBM works in the lid-driven cavity
example.
For the validations, we use the shear wave decay, 
the Couette flow and the Poissuille flow.
The results show that the simulated results
coincide with the analytical solutions except for
the cases where the relaxation term $\omega$ is 
close to either $0$ or $2$.
After the validations, we perform the lid-driven cavity simulation
and the result exhibits similar dynamics when we have a constant Reynolds number
and more noisy turbulence as the Reynolds number becomes larger.
Furthermore, we test the scalability of the LBM in the lid-driven cavity simulation.
The experiments show speedup in all the settings compared to
the serial implementations.
On the other hand, the smaller domains have
less efficiency and they even slow down as the number of processes increases.
This observation corresponds to the intuition from Amdahl's law due to
the latency of the communication and the waiting for the synchronization.
Recall that the parallel implementation is tested by the direct
comparison of the velocity field with the identical settings
as the serial implementation. 
