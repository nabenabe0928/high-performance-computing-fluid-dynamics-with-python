\chapter{Implementation}
\vspace{-5mm}
In this chapter, we describe how the LBM is implemented in Python
and how to compute the LBM in parallel.
All the implementation is assuming that
the physical domain is discretized by D2Q9
and the horizontal axis is $x$ and 
the vertical axis is $y$, respectively.
Since the most important
Note that entire codes are based on
Numpy\footnote{Numpy: https://numpy.org/}
and mpi4py\footnote{mpi4py: https://mpi4py.readthedocs.io/en/stable/}.

\section{Main routine}
Algorithm~\ref{alg:lattice-boltzmann-method-algorithm}
shows the pseudocode of the main processing in the LBM.
Recall that $f(\xv, t)$.shape = $(X, Y, 9)$,
$\rho(\xv, 0)$.shape = $(X, Y)$ and $\uv(\cdot, 0)$.shape = $(X, Y, 2)$, 
Note that the order of each step might vary depending on literature\cite{}.
First, we provide the initial values for the density and the velocity.
Then, we compute the probability function and equillibrium and
apply the collision step.
The equillibrium implementation is shown in Algorithm~\ref{alg:equillibrium-algorithm}.
After applying equillibrium, we perform
streaming operation shown in Algorithm~\ref{alg:streaming-algorithm}
and slide each quantity to the adjacent cells.
Finally, we apply the boundary handling at each boundary cell as 
described in Algorithm~\ref{alg:boundary-conditions-algorithm}
and update the density and the velocity as in Eq~\ref{}.
Note that we use as much slicing as possible in 
Algorithms to speed up the runtime as much as possible.

The Algorithm~\ref{alg:streaming-algorithm} uses
the numpy.roll operation that enables
to handle periodic boundary conditions automatically.
This function rolls the array in the following manner:
\begin{equation}
\begin{aligned}
  \text{np.roll}(f[x][y][i], \text{shift}=\cv_i, \text{axis}=(0, 1)) =
  f[nx][ny][i] \\
  \text{where }
  nx = (x + \cv_i[0]) \% X,
  ny = (y + \cv_i[1]) \% Y
\end{aligned}
\end{equation}
where $i$ is the direction index in D2Q9 and $\cv_i$ is the vector
that specifies the $i$-th direction in D2Q9.
In the Algorithm~\ref{alg:boundary-conditions-algorithm},
we use boolean matrices in\_boundary and out\_boundary
that have the shape of $(X, Y, 9)$ so that 
we can refer to only elements that have bounce back or collision
with the boundary.
Additionally, we compute $\rho_w$ by the average density in Eq~\ref{}.
Note that the domain is extended with virtual nodes 
at both edges of the periodic boundary in the PBC with pressure variation
so that the implementation is more straightforward. 


\begin{algorithm}[tb]
  \caption{The main routine of the lattice Boltzmann method}
  \label{alg:lattice-boltzmann-method-algorithm}
  \begin{algorithmic}[1]
    \Statex{The grid size: $X, Y$,
    Relaxation factor : $\omega$,
    Initial velocity: $\uv_0$,
    Initial density: $\rho_0$
    } \Comment{Inputs}
    \Statex{Boundary conditions}
    \Function{lattice boltzmann method}{}
    \State{$\rho(\xv, 0) = \rho_0, \uv(\xv, 0) = \uv_0$ for all $\xv \in [0, X) \times [0, Y)$}
    \For{$t= 0, 1, \dots$}
    \State{$\feq(\cdot, t),f^\star(\cdot, t)$ = equillibrium($f(\cdot, t), \rho(\cdot, t), \uv(\cdot, t)$)}
    \Comment{Eq~\ref{}}
    \State{$f^\star$ = $f + \omega (\feq - f)$}
    \Comment{Eq~\ref{}}
    \State{$f^\star(\cdot + \cv \dt, t)$ =streaming($f(\cdot, t), \feq(\cdot, t)$)}
    \Comment{Eq~\ref{}}
    \State{$f(\cdot, t + 1)$ = boundary\_handling($f^\star(\cdot, t),\feq(\cdot, t),\text{**kwargs}$)}
    \Comment{Eq~\ref{}}
    \State{$\rho(\cdot, t + 1), \uv(\cdot, t + 1)$=moments\_update($f(\cdot, t + 1)$)}
    \Comment{Eq~\ref{}}
    \EndFor
    \EndFunction
  \end{algorithmic}
\end{algorithm}

\begin{algorithm}[tb]
  \caption{Equillibrium}
  \label{alg:equillibrium-algorithm}
  \begin{algorithmic}[1]
    \Statex{w = $ \text{np.array([}
    \frac{4}{9}, \frac{1}{9}, \frac{1}{9}, 
    \frac{1}{9}, \frac{1}{9}, \frac{1}{36}, 
    \frac{1}{36}, \frac{1}{36}, \frac{1}{36}
    \text{])}$, c = $\cv$ in Eq~\ref{}}
    \Function{equillibrium}{$\rho$ = $\rho(\cdot, t)$, u = $\uv(\cdot, t)$}
    \Comment{$\uv$.shape = $(X, Y, 2)$, $\rho$.shape = $(X, Y)$}
    \State{u\_norm2 = (u ** 2).sum(axis=-1)[..., None]}
    \State{u\_at\_c = u @ c.T}
    \Comment{u\_at\_c.shape = $(X, Y, 9)$}
    \State{w\_tmp, $\rho$\_{tmp} = w[None, None, ...], $\rho$[..., None]}
    \Comment{Adapt the shapes to u\_at\_c}
    \State{$\feq$ = w\_tmp * $\rho$\_tmp * (1 + 3 * u\_at\_c + 4.5 * (u\_at\_c) ** 2)-1.5 * u\_norm2}
    \State{{\bf return} $\feq$}
    \EndFunction
  \end{algorithmic}
\end{algorithm}

\begin{algorithm}[tb]
  \caption{Streaming operation}
  \label{alg:streaming-algorithm}
  \begin{algorithmic}[1]
    \Statex{c = $\cv$ in Eq~\ref{}}
    \Function{streaming}{$f^\star$ = $f^\star(\cdot, t)$}
    \State{$f^{\rm post}$ = np.zeros\_like($f^\star$)}
    \For{$i= 0, 1, \dots, 8$}
    \State{$f^{\rm post}[..., i]$=np.roll($f^\star$[..., i], shift=c[i], axis=(0, 1))}
    \Comment{Slide $f^\star$ one step to c[i]}
    \EndFor
    \State{{\bf return} $f^{\rm post}$}
    \EndFunction
  \end{algorithmic}
\end{algorithm}

\begin{algorithm}[tb]
  \caption{Boundary conditions}
  \label{alg:boundary-conditions-algorithm}
  \begin{algorithmic}[1]
    \Statex{
      Boolean matrix that represents
      where we have the bounce back: in\_boundary
    }
    \Statex{
      Boolean matrix that represents
      where we have the collision: out\_boundary
    }
    \Statex{
      The indices in D2Q9 s.t. the flow comes in
      given boundaries: in\_indices
    }
    \Statex{
      The indices in D2Q9 s.t. the flow goes out
      given boundaries: out\_indices
    }
    \Function{boundary handlling}{$f^\star$ = $f^\star(\cdot, t)$,
      $\feq$ = $\feq(\cdot, t)$}
    \If{Rigid wall}
    \State{$f$[in\_boundary] = $f^\star$[out\_boundary]}
    \EndIf
    \If{Moving wall}
    \State{coef = np.zeros\_like(out\_boundary)}
    \State{{\bf for} out\_idx, ci, wi in zip(out\_indices, c, w) {\bf do}}
    \Indent
    \State{coef[:, :, out\_idx] = 2 * wi * (ci @ $\uv_w$)} / c\_s ** 2
    \EndIndent
    \State{$f$[in\_boundary] = $f^\star$[out\_boundary]-$\rho_w$[out\_boundary] * coef[out\_boundary]}
    \EndIf
    \If{PBC with pressure variation}
    \Comment{fluid flows from $x = 0$ to $X - 1$}
    \State{$\feq_{\rm in}, \feq_{\rm out}$ = equilibrium($\rho_{\rm in}$, $\uv$[-2]), equilibrium($\rho_{\rm out}$, $\uv$[1])}
    \State{$f^{\rm post}$[0][:,out\_indices]=$\feq_{\rm in}$[:,out\_indices]+($f$[-2][:,out\_indices]-$\feq$[-2][:,out\_indices])}
    \State{$f^{\rm post}$[-1][:, in\_indices]=$\feq_{\rm out}$[:, in\_indices]+($f$[1][:, in\_indices] - $\feq$[1][:, in\_indices])}
    \EndIf
    \State{{\bf return} $f$}
    \EndFunction
  \end{algorithmic}
\end{algorithm}

\section{Parallel computation by MPI}
In order to process the LBM in parallel,
we employ the spatial domain decomposition and
the messaging passing interface (MPI)
so that we can compute collision step of the LBM
in parallel.
Note that the collision step does not require
any communication between processes.
When we are provided the number of processes of $P$,
we first factorize $P$ such that $P = P_0 \times P_1$
where $P_0, P_1 \in \mathbb{Z}^{+}$
and $P_0, P_1 = \text{arg} \min_{P_0, P_1}(|| P_1 - P_0 ||)$.
Then, we divide the $x$-axis into $P_0$ intervals and
$y$-axis into $P_1$ intervals where
any pairs of intervals $I_{i}, I_{j}$ in a direction satisfies
$-1 \leq || I_{i} || - ||I_{j}|| \leq 1$.
Note that $||I||$ is the size of the interval $I$.
This split of the domain achieves the most balanced distribution of
the computation.
For the streaming step, we need to consider particles
moving from one process to another.
We implement it using so-called {\bf ghost cells}
around the actual computational domain.
Figure~\ref{} shows the conceptual visualization of
how each process communicate and ghost cells work.
Since each process requires the four edges of adjacent
processes, the commnications are required four times
for each process.
The Algorithm~\ref{alg:mpi-algorithm}
shows the implementation using mpi4py.
grid\_manager is our self-developed module
that allows to get useful information related to
process location, the adjacent relation and so on.
Sendrecv function is used for the commnication and
each process receives an array from dest that is sent
by neighbor and sends sendbuf to neighbor.
Note that buf is the abbreviation of buffer and
it is literally used for the buffer to communicate data.


\begin{algorithm}[tb]
  \caption{The communication of
  the particle probability density function}
  \label{alg:mpi-algorithm}
  \begin{algorithmic}[1]
    \Statex{Process and lattice grids management: grid\_manager}
    \Function{communication}{}
    \State{{\bf for} dir in grid\_manager.neighbor\_directions {\bf do}}
    \Comment{Iterate over the D2Q9 index}
    \Indent
    \State{dx, dy = $\cv_i$}
    \State{sendidx = grid\_manager.step\_to\_idx(dx, dy, send=True)}
    \State{recvidx = grid\_manager.step\_to\_idx(dx, dy, send=False)}
    \State{neighbor = grid\_manager.get\_neighbor\_rank(dir)}
    \If{dx == 0} \Comment{send to top and bottom}
    \State{sendbuf = $f$[:, sendidx, ...].copy()}
    \State{grid\_manager.rank\_grid.Sendrecv(sendbuf=sendbuf, dest=neighbor,}
    \State{\hspace{56mm} recvbuf=recvbuf, source=neighbor)}
    \State{$f$[:, recvidx, ...] = recvbuf}
    \ElsIf{dy == 0} \Comment{send to left and right}
    \State{sendbuf = $f$[sendidx, ...].copy()}
    \State{grid\_manager.rank\_grid.Sendrecv(sendbuf=sendbuf, dest=neighbor,}
    \State{\hspace{56mm} recvbuf=recvbuf, source=neighbor)}
    \State{$f$[recvidx, ...] = recvbuf}
    \Else
    \State{sendbuf = $f$[sendidx[0], sendidx[1] ...].copy()}
    \State{grid\_manager.rank\_grid.Sendrecv(sendbuf=sendbuf, dest=neighbor,}
    \State{\hspace{56mm} recvbuf=recvbuf, source=neighbor)}
    \State{$f$[recvidx[0], recvidx[1], ...] = recvbuf}
    \EndIf
    \EndIndent
    \State{{\bf return} $f$}
    \EndFunction
  \end{algorithmic}
\end{algorithm}